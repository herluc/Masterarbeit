\documentclass[a4paper,12pt,arial, rgb]{tubsartcl}
\usepackage[utf8]{inputenc}
\usepackage[ngerman]{babel}
\usepackage{color, parskip}
\defbglayout[sender=top,pages=all]{newlayout}{%
\showtubslogo[left]
%das Institutslogo - hier möglicherweise einen neuen Pfad eingeben
\showlogo{\includegraphics{../InA-Logo-rgb}}}
\definecolor{InAGreen}{RGB}{171, 193, 59} 
%los gehts...
\begin{document}
%%%%%%%%%%%%%%%%%%%%%%% Deckblatt einer studentischen Arbeit ohne Bild %%%%%%%%%%%%%%%%%%%
\begin{gausspage}
\usebglayout{newlayout}
\begin{segment}[c,bgcolor=tubsWhite]{2}	
%%%%%%%%%%%%%%%%%%%%%%%%%%%%%%% hier kommt ein Titel: %%%%%%%%%%%%%%%%%%%%%%%%%%%%%%%%%%%%
\headline{A Statistical Approach for the Fusion of Data and
Finite Element Analysis in Vibroacoustics}
\end{segment}
\begin{segment}[c,bgcolor=tubsWhite]{2}	
%%%%%%%%%%%%%%%%%%%%%%%%%%% Segment für ein optionales Bild %%%%%%%%%%%%%%%%%%%%%%%%%%%%%%
	% wenn Bild gewünscht, dann in Zeile 19 "bgcolor=tubsWhite" gegen
	% "bgimage = <<pfad zum Bild>>" ersetzen.
\end{segment}
\begin{segment}[c,bgcolor=InAGreen,fgcolor=tubsWhite]{3}					\large \noindent
%%%%%%%%%%%%%%%%%%%%%%%%%%% und hier kommen Daten zur Arbeit %%%%%%%%%%%%%%%%%%%%%%%%%%%%%%
\section*{\textcolor{tubsWhite}{Masterarbeit}} 	
\textbf{Lucas Hermann}

Matr.-Nr.: 4990987



\textbf{Erstprüferin:} Prof. Dr.-Ing. Sabine C. Langer\\
\textbf{Zweitprüfer:}  $\;$Prof. Dr.-Ing. Ulrich Römer
\end{segment}
\begin{segment}[innerpadding=vnone,bgcolor=tubsGreen,%
	fgcolor=tubsWhite]{1} \vskip-3pt \large%
{\small
30. September 2021
}
\end{segment}
\end{gausspage}
\end{document}
